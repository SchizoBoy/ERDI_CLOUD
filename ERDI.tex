\documentclass[a4paper, 12pt]{article}
\usepackage[utf8]{inputenc}
\usepackage[T1]{fontenc}
\usepackage[scale=.8]{geometry}
\usepackage[french]{babel}
\usepackage{hyperref}
\usepackage{graphicx}
\usepackage{amsmath}
\usepackage{amssymb}

\title{Sécurité des données dans le cloud\\Doit-on se méfier du cloud ?}
\author{M1 Informatique}
\date{}

\begin{document}
  \maketitle
  \newpage
  \tableofcontents
  \listoffigures
  \newpage

  \section{Introduction}
    Aujourd'hui, une majorité de nos données personnelles, professionnelles,
    publiques comme privées, transitent par des services de stockage en ligne,
    des données qui peuvent aller de simples images publiques à des données
    confidentielles comme par exemple des données bancaires ou encore des
    documents secrets pouvant impacter l'intégralité d'un pays ou même du monde.

    Il est impératif de s'assurer de la sécurité de ses services de stockage,
    car les conséquences découlant de possibles négligeances pourraient avoir
    des effets catastrophiques.

    \subsection{Quelques statistiques}
      Un bon moyen de visualiser l'ampleur que le cloud à prit dans le monde est
      de l'illustrer par quelques statistiques
      \begin{itemize}
        \item Environ 94\% des entreprises utilisent des services de cloud.
        \item En 2025, selon les prévisions actuelles, 100 zettabytes de données
              devraient être stockées sur des services en ligne
              \footnote{Cela représente 10$^{23}$ octets de données soit
              10$^{11}$ disques dur de 1To}.
        \item En 2013, au total, seulement un exabyte de données était stocké
              sur des services de cloud\footnote{100 zettabytes = 100000
              exabytes}.
      \end{itemize}

      On peut voir par ces quelques statistiques que non seulement le nombre de
      données stockées en ligne est gigantesque, mais aussi qu'il croit très
      rapidement, il est donc nécessaire de se prémunir contre les risques
      potentiellement encourus.

    \subsection{Différents modèles de cloud}
      Les fournisseurs de services de cloud peuvent proposer différents modèles,
      proposant différents prestations que l'entreprise gère ou non pour
      l'utilisateur, les trois modèles généralement proposés sont les service
      d'infrastructure, de plateforme et de logiciel, respectivement nommés
      IaaS, PaaS et SaaS\footnote{aaS signifie as-a-Service, cette mention
      signifie qu'un fournisseur prend en charge la gestion du service et de
      certaines fonctionnalités pour les utilisateurs}. La différence entre ces
      trois modèles réside en le nombre de fonctionnalité dont le fournisseur
      laisse la gestion à l'utilisateur, le modèle IaaS est celui qui laisse le
      plus de fonctionnalités à la gestion de l'utilisateur, contrairement au
      modèle SaaS qui lui est géré complètement par le fournisseur. Néanmoins,
      aucun des modèles ne laisse au client la possibilité de gérer les
      fonctionnalités concernant l'accès réseau et le système de stockage
      utilisé.

  \section{Les failles de sécurité qui peuvent mener à un vol de donnée}
    L'inquiétude suscitée par le cloud est que ce genre de service hébèrge
    énormément d'utilisateurs. Si un attaquant réussi à pénétrer le cloud, il
    met la main sur les données de l'intégralité des personnes qui en font
    l'usage. Nous allons donc voir ce qui pourrait permettre à un attaquant de
    pouvoir voler les données stockées en ligne dans ce type de service.

    \subsection{La gestion de la mémoire dans le cloud}
      Le cloud est comme un disque dur géant, et à l'image d'un disque dur de
      taille normale partagé par plusieurs systèmes et/ou utilisateurs, il est
      important de se concentrer sur la gestion de l'espace alloué à chacun.

      Un des problèmes cité page 3/4 de cette étude [VR2016] est le fait que la
      mémoire effacée peut être récupérée. En effet, \textit{"The resource
      allocated to a particular user may be assigned to the other user at some
      later point of time"}\footnote{Traduction : l'espace alloué à un
      utilisateur peut être assigné à un autre à un autre moment}. Les services
      de cloud marchent nécessairement avec un système d'allocation de mémoire
      dynamique en fonction de la requête d'espace de l'utilisateur. Les
      services n'ayant pas une quantité infinie de stockage, Un morceau de
      l'espace va surement être utilisé successivement par plusieurs
      utilisateurs différents. L'utilisateur de ce morceau de l'espace pourra
      donc potentiellement récupérer d'anciennes données effacées qui ne lui
      appartiennent pas.

      En effet, sur la plupart des machines, il est possible de récupérer des
      données qui ont été effacés. Les données ne sont en réalité pas rééllement
      effacées tant que le disque ou le dispositif de stockage n'a pas réécrit
      d'autres données par dessus. Si un utilisateur dispose d'un morceau de la
      plage mémoire allouée précedémment à un autre utilisateur, il pourra donc
      tenter de les récupérer. \\

      Un autre problème posé par le partage massif du cloud est la fragmentation
      de la mémoire. En effet, un problème pourraient résider dans la mauvaise
      ségmentation de la mémoire. Ceci pourrait permettre l'accès au même
      morceau de mémoire par plusieurs personnes, et un utilisateur pourrait
      voler des données à un autre, ou même récupérer des données supprimées
      comme dit plus haut.

    \subsection{Risques physiques}
      Les clouds sont des espaces de stockages, il est aussi nécessaire de
      considérer les menaces liées aux supports de stockages physiques des
      données comme les centres de stockages. Un attaquant pourrait très bien
      décider de pirater physiquement un centre de données, il est donc aussi
      nécessaire de se prémunir des risques physiques liées à l'intégrité des
      données stockées dans le cloud.

  \section{Anticipation et prévention des risques}

  \section{L'avenir du cloud du point de vue de la sécurité informatique}

  \section{Conclusion}

  \section{Annexes}

  \section{Bibliographie}
    \begin{itemize}
      \item \href{https://www.cloudwards.net/cloud-computing-statistics/}{Quelques statistiques à propos du cloud} : \url{https://www.cloudwards.net}
      \item \href{https://parachute.cloud/cloud-computing-statistics/}{Différentes statistiques sur l'utilisation et la présence du cloud} : \url{https://parachute.cloud/}
      \item \href{https://www.globaldots.com/resources/blog/how-much-is-stored-in-the-cloud/}{Données datant de 2013 sur la taille du cloud} : \url{https://www.globaldots.com}
      \item \href{https://aws.amazon.com/fr/types-of-cloud-computing/}{Différents types de cloud et de services proposés} : \url{https://aws.amazon.com}
      \item \href{https://www.redhat.com/fr/topics/cloud-computing/what-is-iaas}{Le type de service IaaS et comparaisons avec PaaS et SaaS} : \url{https://www.redhat.com/}
      \item \href{https://www.javatpoint.com/cloud-deployment-model}{Différents types de déploiement des services de cloud} : \url{https://www.javatpoint.com/}
      \item \href{https://www.cyberuniversity.com/post/la-securite-dans-le-cloud-principaux-risques-et-challenges}{Principaux risques encourus en stockant sur le cloud} : \url{https://www.cyberuniversity.com/}
      \item \href{https://www.sciencedirect.com/science/article/pii/S1877050916315812}{[VR2016] A Study on Data Storage Security Issues in Cloud Computing} : \url{https://www.sciencedirect.com/}
      \item \href{https://www.techtarget.com/searchdisasterrecovery/definition/data-recovery}{Récupération de données effacées} : \url{https://www.techtarget.com/}
      \item \href{https://www.runtime.org/recoverability.htm}{Autres données sur la récupération de données effacées} : \url{https://www.runtime.org/}
      \item \href{https://www.techniques-ingenieur.fr/actualite/articles/la-securite-dans-le-cloud-une-approche-fournisseur-basee-sur-les-risques-15550/}{Différents risques liés cloud} : \url{https://www.techniques-ingenieur.fr/}
    \end{itemize}

\end{document}
