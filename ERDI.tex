\documentclass[a4paper, 12pt]{article}
\usepackage[utf8]{inputenc}
\usepackage[T1]{fontenc}
\usepackage[scale=.8]{geometry}
\usepackage[french]{babel}
\usepackage{hyperref}
\usepackage{graphicx}
\usepackage{amsmath}
\usepackage{amssymb}

\title{Sécurité des données dans le cloud\\Doit-on se méfier du cloud ?}
\author{M1 Informatique}
\date{}

\begin{document}
  \maketitle
  \newpage
  \tableofcontents
  \newpage

  \section{Introduction}
    Aujourd'hui, une majorité de nos données personnelles, professionnelles,
    publiques comme privées, transitent par des services de stockage en ligne,
    des données qui peuvent aller de simples images publiques à des données
    confidentielles comme par exemple des données bancaires ou encore des
    documents secrets pouvant impacter l'intégralité d'un pays ou même du monde.

    Il est impératif de s'assurer de la sécurité de ses services de stockage,
    car les conséquences découlant de possibles négligeances pourraient avoir
    des effets catastrophiques.

    \subsection{Quelques statistiques}
      Un bon moyen de visualiser l'ampleur que le cloud à prit dans le monde est
      de l'illustrer par quelques statistiques
      \begin{itemize}
        \item Environ 94\% des entreprises utilisent des services de cloud.
        \item En 2025, selon les prévisions actuelles, 100 zettabytes de données
              devraient être stockées sur des services en ligne
              \footnote{Cela représente 10$^{23}$ octets de données soit
              10$^{11}$ disques dur de 1To}.
        \item En 2013, au total, seulement un exabyte de données était stocké
              sur des services de cloud\footnote{100 zettabytes = 100000
              exabytes}.
      \end{itemize}

      On peut voir par ces quelques statistiques que non seulement le nombre de
      données stockées en ligne est gigantesque, mais aussi qu'il croit très
      rapidement, il est donc nécessaire de se prémunir contre les risques
      potentiellement encourus.

    \subsection{Différents modèles de cloud}
      Les fournisseurs de services de cloud peuvent proposer différents modèles,
      proposant différents prestations que l'entreprise gère ou non pour
      l'utilisateur, les trois modèles généralement proposés sont les service
      d'infrastructure, de plateforme et de logiciel, respectivement nommés
      IaaS, PaaS et SaaS\footnote{aaS signifie as-a-Service, cette mention
      signifie qu'un fournisseur prend en charge la gestion du service et de
      certaines fonctionnalités pour les utilisateurs}. La différence entre ces
      trois modèles réside en le nombre de fonctionnalité dont le fournisseur
      laisse la gestion à l'utilisateur, le modèle IaaS est celui qui laisse le
      plus de fonctionnalités à la gestion de l'utilisateur, contrairement au
      modèle SaaS qui lui est géré complètement par le fournisseur. Néanmoins,
      aucun des modèles ne laisse au client la possibilité de gérer les
      fonctionnalités concernant l'accès réseau et le système de stockage
      utilisé.

  \section{Les failles de sécurité qui peuvent mener à un vol de données}
    L'inquiétude suscitée par le cloud est que ce genre de service hébèrge
    énormément d'utilisateurs. Si un attaquant réussi à pénétrer le cloud, il
    met la main sur les données de l'intégralité des personnes qui en font
    l'usage. Nous allons donc voir ce qui pourrait permettre à un attaquant de
    pouvoir voler les données stockées en ligne dans ce type de service.

    \subsection{Erreurs humaines et contrôle d'accès}
      La grande majorité des failles qui ont mené à des vols de données sur 
      un cloud proviennent d'une erreur humaine, on peut citer parmis elles les
      défauts de configuration. c'est le modèle IaaS qui est mis en cause dans
      la plupart des cas. En effet, ce modèle de cloud est très flexible et
      permet aux utilisateurs de pouvoir configurer et gérer de nombreux aspects
      du cloud qu'ils utilisent, en les configurant potentiellement d'une façon
      trop peu sécurisée. Selon l'étude [AF2019], \textit{about 99\% of
      misconfigurations go unnoticed by companies using IaaS}
      \footnote{Traduction : Environ 99\% des erreurs de configurations ne sont
      pas remarquées par les entreprises utilisant le modèle IaaS}. \\

      Des erreurs peuvent aussi être commises losqu'il s'agit de l'accès des
      utilisateurs aux services et aux données stockées. Parfois un trop grand
      nombre de personne peut avoir accès à des données sensibles, par exemple
      au sein d'une entreprise, sans pour autant que cela soit nécessaire, cela
      découlant d'un manque de contrôle de ces accès. Laisser à trop de
      personnes l'accès à des données sensibles multiplie le risque d'attaque.
      On peut imaginer un attaquant extérieur qui pourrait vouloir pénétrer le
      stockage par ricochet en utilisant les utilisateurs concernés, mais aussi
      les utilisateurs eux-mêmes qui pourrait cacher un attaquant interne, il
      est donc nécessaire d'assurer un contrôle des accès rigoureux.

    \subsection{La gestion de la mémoire dans le cloud}
      À l'image d'un disque dur de taille normale, le cloud est un espace de
      stockage partagé par plusieurs systèmes et/ou utilisateurs, il est
      important de se concentrer sur la gestion de l'espace alloué à chacun.

      Un des problèmes cité page 3/4 de cette étude [VR2016] est le fait que la
      mémoire effacée peut être récupérée. En effet, \textit{"The resource
      allocated to a particular user may be assigned to the other user at some
      later point of time"}\footnote{Traduction : l'espace alloué à un
      utilisateur peut être assigné à un autre à un autre moment}. Les services
      de cloud marchent nécessairement avec un système d'allocation de mémoire
      dynamique en fonction de la requête d'espace de l'utilisateur. Les
      services n'ayant pas une quantité infinie de stockage, Un morceau de
      l'espace va surement être utilisé successivement par plusieurs
      utilisateurs différents. L'utilisateur de ce morceau de l'espace pourra
      donc potentiellement récupérer d'anciennes données effacées qui ne lui
      appartiennent pas.

      En effet, sur la plupart des machines, il est possible de récupérer des
      données qui ont été effacés. Les données ne sont en réalité pas rééllement
      effacées tant que le disque ou le dispositif de stockage n'a pas réécrit
      d'autres données par dessus. Si un utilisateur dispose d'un morceau de la
      plage mémoire allouée précedémment à un autre utilisateur, il pourra donc
      tenter de les récupérer. \\

      Un autre problème posé par le partage massif du cloud est la fragmentation
      de la mémoire. En effet, un problème pourraient résider dans la mauvaise
      ségmentation de la mémoire. Ceci pourrait permettre l'accès au même
      morceau de mémoire par plusieurs personnes, et un utilisateur pourrait
      voler des données à un autre, ou même récupérer des données supprimées
      comme dit plus haut.

    \subsection{Risques physiques}
      Les clouds sont des espaces de stockages, il est aussi nécessaire de
      considérer les menaces liées aux supports de stockages physiques des
      données comme les centres de stockages. Un attaquant pourrait très bien
      décider de pirater physiquement un centre de données, il est donc aussi
      nécessaire de se prémunir des risques physiques liées à l'intégrité des
      données stockées dans le cloud.

  \section{Anticipation et prévention des risques}
      L'identification des risques liés à une installation dans le cloud est
      primordiale car elle permet de pouvoir réfléchir à des solutions pour 
      anticiper les attaques et les solutionner. Puisqu'il existe de nombreux
      risques dont une partie on été traités dans la partie précédente, il 
      existe un grand nombre de méthodes et de solutions pour les anticiper
      et les éviter.

    \subsection{Sur le plan physique}
      Les clouds étant des espaces de stockages numériques auxquels un 
      utilisateur à très peu voir pas du tout d'accès sur le plan physique. 
      Qu'il soit un particulier ou une entreprise, lorsqu'un utilisateur 
      prévoit l'utilisation d'un service cloud, il devient alors nécessaire de
      procéder à une étude des risques physiques liés au fournisseur de 
      ce service et à la politique de sécurité qu'il applique au sein de ses 
      infrastructures. Y a-t-il de la surveillance, des contrôle d'accès ou 
      d'autres mesures permettant de garantir un accès limité au site ? Il
      existe notamment certaines certifications qui peuvent être utilisées dans
      ce but comme la certification ISO 9000\footnote{Certification délivrée 
      par l'organisme ISO selon un certains nombres de critères comme 
      l'organisation, les clients, les données, les systèmes etc.} ou SAS 70
      \footnote{State on Auditing Standards no. 70, remplacée par SSAE16. Il 
      s'agit d'une certification spécialisée pour les centre de données} par 
      exemple. \\

      Tout cela étant dit, une perte de donnée ou une rupture d'accès à un 
      service dans le cloud ne provient pas toujours d'une action volontaire 
      d'un ou plusieurs individus. C'est pour cela que le contrôle d'accès et 
      la surveillance ne suffisent pas à garantir la sécurité des données ou
      du service. Il est aussi nécessaire de se prémunir des risques liés à des
      catastrophes naturelles comme les incendies, les inondations ou bien 
      plus simplement des coupures de courant par exemple. Cela peut-être 
      effectué en prévoyant une redondance des données grâce à des sauvegardes 
      régulières chiffrées et stockées dans d'autres infrastructures. Dans le 
      cadre d'un service qui outrepasse le simple stockage de données, il est 
      aussi nécessaire de prévoir une redirection du service vers une autre 
      plateforme fonctionnelle. Cela parmet d'éviter les interruptions de 
      service pour des applications cloud qui y seraient sensibles.
      
    \subsection{Sur le plan technique}
      Afin de permettre une sécurisation du cloud, il est nécessaire de 
      s'intéresser à l'aspect technique. Par aspect technique, on entend ici
      le côté interne aux systèmes et à leur configuration. Que ce soit au 
      niveau de la configuration système ou logicielle ou éventuellement grâce 
      à des logiciels de sécurité clé en main centralisant et rassemblant 
      diverses solutions de sécurité. \\

      Au niveau des solutions techniques, tous les types de clouds ne sont pas
      logés à la même enseigne. En effet, un utilisateur de cloud de type SaaS
      n'aura pas les mêmes responsabilités et les mêmes obligations qu'un
      utilisateur de cloud de type IaaS par exemple. Quoiqu'il en soit, les
      mêmes solutions seront finalement implantés sur le service, il s'agit
      uniquement de comprendre que les tâches seront partagées entre le 
      fournisseur de services et le client. \\

      Le premier niveau de sécurité nécessaire à implanter est un système 
      d'authentification sécurisé. Cela passe évidemment par l'utilisation 
      de mots de passes suffisemment sûrs avec pourquoi pas une expiration
      afin de forcer le ou les utilisateurs à le changer régulièrement. Il 
      s'agit d'une pratique courante dans le monde de l'entreprise qui a tout
      autant de sens au sein d'une installation  cloud. On peut aussi utiliser 
      de l'authentification multi-factorielle afin de renforcer d'autant plus 
      la sécurité d'accès au service. Ce point sur la sécurisation de l'accès 
      est crucial sur une installation de ce type puisque l'ensemble des 
      vulnérabilités du cloud se situe au sein d'une seule infrastructure
      qui plus est connue de tous. Et c'est bien là sa particularité. \\

    \subsection{Un aspect social}
      Le domaine de l'ingénieurie sociale est en hausse car il s'agit très 
      certainement de la méthode la plus simple et non moins efficace pour 
      obtenir l'accès à des données ou des services illégitimement. Il est 
      crucial pour quiconque s'intéresse à la technologie du cloud et d'autant
      plus pour les entreprises de se former et de former l'ensemble des 
      équipes au risques bien réels auxquels ils peuvent faire face dans leur 
      quotidien. \\
      
      Pour une entreprise du cloud il est aussi nécessaire de définir 
      clairement des règles de conduite pour l'ensemble du personnel afin
      d'éviter au maximum les failles de sécurité sensibles à l'ingénieurie
      sociale.

  \section{L'avenir de la sécurité dans le cloud avec l'IA ?}
    \subsection{La prévention par l'IA}
    \subsection{La protection par l'IA}
    \subsection{Comment ça marche ?}
    \subsection{Solution viable ou non ?}

  \section{Conclusion}

  \section{Annexes}

  \section{Bibliographie}
    \begin{itemize}
      \item \href{https://www.cloudwards.net/cloud-computing-statistics/}{Quelques statistiques à propos du cloud} : \url{https://www.cloudwards.net}
      \item \href{https://parachute.cloud/cloud-computing-statistics/}{Différentes statistiques sur l'utilisation et la présence du cloud} : \url{https://parachute.cloud/}
      \item \href{https://www.globaldots.com/resources/blog/how-much-is-stored-in-the-cloud/}{Données datant de 2013 sur la taille du cloud} : \url{https://www.globaldots.com}
      \item \href{https://aws.amazon.com/fr/types-of-cloud-computing/}{Différents types de cloud et de services proposés} : \url{https://aws.amazon.com}
      \item \href{https://www.redhat.com/fr/topics/cloud-computing/what-is-iaas}{Le type de service IaaS et comparaisons avec PaaS et SaaS} : \url{https://www.redhat.com/}
      \item \href{https://www.javatpoint.com/cloud-deployment-model}{Différents types de déploiement des services de cloud} : \url{https://www.javatpoint.com/}
      \item \href{https://www.cyberuniversity.com/post/la-securite-dans-le-cloud-principaux-risques-et-challenges}{Principaux risques encourus en stockant sur le cloud} : \url{https://www.cyberuniversity.com/}
      \item \href{https://www.sciencedirect.com/science/article/pii/S1877050916315812}{[VR2016] A Study on Data Storage Security Issues in Cloud Computing} : \url{https://www.sciencedirect.com/}
      \item \href{https://www.techtarget.com/searchdisasterrecovery/definition/data-recovery}{Récupération de données effacées} : \url{https://www.techtarget.com/}
      \item \href{https://www.runtime.org/recoverability.htm}{Autres données sur la récupération de données effacées} : \url{https://www.runtime.org/}
      \item \href{https://www.techniques-ingenieur.fr/actualite/articles/la-securite-dans-le-cloud-une-approche-fournisseur-basee-sur-les-risques-15550/}{Différents risques liés cloud} : \url{https://www.techniques-ingenieur.fr/}
      \item \href{https://www.magic.fr/cloud-public-les-erreurs-de-configuration-sont-extremement-frequentes/}{Erreurs humaines, défauts de configuration} : \url{https://www.magic.fr/}
      \item \href{https://www.mcafee.com/enterprise/en-us/assets/reports/restricted/rp-cloud-adoption-risk-report-iaas.pdf}{[AF2019] Cloud-Native: The Infrastructure-as-a-Service
(IaaS) Adoption and Risk Report} : \url{https://www.mcafee.com/}
    \end{itemize}

\end{document}
