\documentclass[a4paper, 12pt]{article}
\usepackage[utf8]{inputenc}
\usepackage[T1]{fontenc}
\usepackage[scale=.9]{geometry}
\usepackage{hyperref}
\usepackage{graphicx}
\usepackage{amsmath}
\usepackage{amssymb}

\title{Sécurité des données dans le cloud\\Doit-on se méfier du cloud ?}
\author{M1 Informatique}
\date{}

\begin{document}
  \maketitle
  \newpage
  \tableofcontents
  \listoffigures
  \newpage

  \section{Introduction}
    Aujourd'hui, une majorité de nos données personnelles, professionnelles,
    publiques comme privées, transitent par des services de stockage en ligne,
    des données qui peuvent aller de simples images publiques à des données
    confidentielles comme par exemple des données bancaires ou encore des
    documents secrets pouvant impacter l'intégralité d'un pays ou même du monde.

    Il est impératif de s'assurer de la sécurité de ses services de stockage,
    car les conséquences découlant de possibles négligeances pourraient avoir
    des effets catastrophiques.

    \subsection{Quelques statistiques}
      Un très bon moyen de visualiser l'ampleur que le cloud à prit dans le
      monde est de l'illustrer par quelques statistiques
      \begin{itemize}
        \item Environ 94\% des entreprises utilisent des services de cloud.
        \item En 2025, selon les prévisions actuelles, 100 zettabytes de données
              devraient être stockées sur des services en ligne
              \footnote{Cela représente 10$^{23}$ octets de données soit
              10$^{11}$ disques dur de 1To}.
        \item En 2013, au total, seulement un exabyte de données était stocké
              sur des services de cloud\footnote{100 zettabytes = 100000 exabytes}.
      \end{itemize}

      On peut voir par ces quelques statistiques que non seulement le nombre de
      données stockées en ligne est gigantesque, mais aussi qu'il croit très
      rapidement, il est donc nécessaire de se prémunir contre les risques
      potentiellement encourus.

    \subsection{Différents modèles de cloud}
      Les fournisseurs de services de cloud peuvent proposer différents modèles,
      proposant différents prestations que l'entreprise gère ou non pour
      l'utilisateur, les trois modèles généralement proposés sont les service
      d'infrastructure, de plateforme et de logiciel, respectivement nommés
      IaaS, PaaS et SaaS\footnote{aaS signifie as-a-Service, cette mention
      signifie qu'un fournisseur prend en charge la gestion du service et de
      certaines fonctionnalités pour les utilisateurs}. La différence entre ces
      trois modèles réside en le nombre de fonctionnalité dont le fournisseur
      laisse la gestion à l'utilisateur, le modèle IaaS est celui qui laisse le
      plus de fonctionnalités à la gestion de l'utilisateur, contrairement au
      modèle SaaS qui lui est géré complètement par le fournisseur. Néanmoins,
      aucun des modèles ne laisse au client la possibilité de gérer les
      fonctionnalités concernant l'accès réseau et le système de stockage
      utilisé.

    \subsection{Différents types de déploiements}
      Les services de cloud peuvent être déployés de plusieurs manières

  \section{Les risques de l'utilisation de services de cloud}

  \section{Anticipation et prévention des risques}

  \section{L'avenir du cloud du point de vue de la sécurité informatique}

  \section{Conclusion}

  \section{Annexes}
    \begin{figure}

    \end{figure}

  \section{Bibliographie}
    \begin{itemize}
      \item \href{https://www.cloudwards.net/cloud-computing-statistics/}{Quelques statistiques à propos du cloud} : \url{https://www.cloudwards.net}
      \item \href{https://parachute.cloud/cloud-computing-statistics/}{Différentes statistiques sur l'utilisation et la présence du cloud} : \url{https://parachute.cloud/}
      \item \href{https://www.globaldots.com/resources/blog/how-much-is-stored-in-the-cloud/}{Données datant de 2013 sur la taille du cloud} : \url{https://www.globaldots.com}
      \item \href{https://aws.amazon.com/fr/types-of-cloud-computing/}{Différents types de cloud et de services proposés} : \url{https://aws.amazon.com}
      \item \href{https://www.redhat.com/fr/topics/cloud-computing/what-is-iaas}{Le type de service IaaS et comparaisons avec PaaS et SaaS} : \url{https://www.redhat.com/}
      \item \href{https://www.javatpoint.com/cloud-deployment-model}{Différents types de déploiement des services de cloud} : \url{https://www.javatpoint.com/}
    \end{itemize}

\end{document}
